\documentclass[ignoreonframetext,unicode]{beamer}

\usepackage[utf8]{inputenc}
\usepackage[T1]{fontenc}
\usepackage[english,russian]{babel}
\usepackage{cmap}
\usepackage{amsmath}
\usepackage{amsfonts}
\usepackage{amssymb}
\usepackage{graphicx,pgf}
\usepackage{multimedia}

\graphicspath{{./figures/}{../report/figures/}}

\usetheme{Warsaw}

\useinnertheme{circles}   %внутренняя тема
%\useoutertheme{smoothbars}   %внешняя тема
\usecolortheme{seahorse}     %цветовая схема
%\usefonttheme{serif}    %шрифты
%\defbeamertemplate*{footline}{shadow theme}
%\setbeameroption{hide notes}

%номера слайдов
\newcommand*\oldmacro{}%
\let\oldmacro\insertshorttitle%
\renewcommand*\insertshorttitle{%
	\oldmacro\hfill%
	\insertframenumber\,/\,\inserttotalframenumber}
\RequirePackage{caption}
\DeclareCaptionLabelSeparator{defffis}{ }
\captionsetup{justification=centering,labelsep=defffis}

\institute[каф. Прикладная математика ФН-2]{группа ФН2-42Б}
\date{\today}
\titlegraphic{\includegraphics[width=2cm]{emblema.pdf}}
%\renewcommand{\vec}[1]{\text{\mathversion{bold}${#1}$}}

\title[Задача об ограниченном ранце]{Задача об ограниченном ранце. Метод ветвей и~границ, динамическое программирование}
\author[Абрамов З.\,И., Швецов Г.\,A.]{Абрамов З.\,И.\and\\[0.5mm] Швецов Г.\,А.}

\frenchspacing
\righthyphenmin=2

\usepackage{comment}

\begin{comment}
	Красивый блок с заголовком title (если он пустой, то заголовка не будет)
	\begin{block}{title}
		содержимое...
	\end{block}
\end{comment}

\begin{comment}
	Колонки
	https://latex-beamer.com/tutorials/columns/
	
	\begin{columns}
		\begin{column}{0.5\textwidth}
			содержимое...
		\end{column}
		
		\begin{column}{0.5\textwidth}
			содержимое...
		\end{column}
	\end{columns}
	
	Отступ можно сделать добавив третью колонку между ними (тут же можно сделать разделитель):
	\begin{column}{0.01\textwidth}
		%\rule{.1mm}{0.7\textheight}
	\end{column}
	
\end{comment}

\begin{comment}
	Другие команды:
	\begin{center}
		содержимое...
	\end{center}
	\includegraphics[width=0.5\textwidth]{image_name.png}
	
	Какие-то команды:
	\frametitle{explanation}
\end{comment}

% Выравнивание по ширине
\usepackage{ragged2e}
\justifying

\parindent=0.5cm

%%%%%%%%%%%%%%%%%%%%%%%%%%%%%%%%%%%%%%%%%%%%%%%%%%%%%%%%%%%%
%%%%%%%%%%%%%%%%%%%%%%%%%%%%%%%%%%%%%%%%%%%%%%%%%%%%%%%%%%%%
%%%%%%%%%%%%%%%%%%%%%%%%%%%%%%%%%%%%%%%%%%%%%%%%%%%%%%%%%%%%

\begin{document}
	\begin{frame}[plain]
		\maketitle
	\end{frame}

	\begin{frame}{Постановка задачи}
		Задача об ограниченном рюкзаке формулируется следующим образом:
		
		Пусть имеется $n$ типов предметов. Каждый тип предмета $i$ характеризуется весом $w_i$ и стоимостью $c_i$ одного предмета и количеством предметов $k_i$ данного типа. Также имеется рюкзак вместимости $W$.
		
		Требуется собрать набор с максимальной полезностью таким образом, чтобы он имел вместимость не больше $W$. При этом количество предметов типа $i$ не должно превышать $k_i$.
		
		В математической форме:
		\begin{gather*}
			\sum_{i=1}^{n} c_i x_i \to \max \\
			\sum_{i=1}^{n} w_i x_i \leqslant W \\
			\forall i\in\{1,\dots,n\} \quad x_i \in \{0,\dots, k_i\}
		\end{gather*}
	\end{frame}

	\begin{frame}{Метод ветвей и границ}
		содержимое...
	\end{frame}

	% TODO другие слайды

	\begin{frame}{Динамическое программирование}
		содержимое...
	\end{frame}

	% TODO другие слайды

	\begin{frame}{Пример работы}
		% TODO пара примеров, один с одним решением, второй - с двумя, чтобы показать, что метод ветвей и границ ищет все решения, а ДП - только одно
		содержимое...
	\end{frame}

	% TODO время исполнения зависит еще от максимального веса
	\begin{frame}{Сравнение времени исполнения на C++}
		содержимое...
	\end{frame}

	\begin{frame}{Сравнение времени исполнения на Wolfram Mathematica}
		содержимое...
	\end{frame}

	% TODO номально оформить или удалить
	\begin{frame}
		Спасибо за внимание
	\end{frame}

\end{document}